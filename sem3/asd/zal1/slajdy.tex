\documentclass[12pt]{beamer}

\usepackage[polish]{babel}
\usepackage[T1]{fontenc}
\usepackage[utf8]{inputenc}
\usepackage{mathtools}                       % matma
\usepackage{amsfonts,amsmath,amssymb,amsthm} % matma
\usepackage{listings}                        % załączanie kodu źródłowego
\usepackage{tikz}                            % rysowanie i grafy https://tikz.dev/tikz

\title{Komitet strajkowy}
\author{Stanisław Bitner}
\date{listopad 2022}

\usetheme{Warsaw}
\usecolortheme{orchid}

\setbeamertemplate{footline}[Warsaw]
\setbeamertemplate{headline}{}
\setbeamertemplate{page number in head/foot}[totalframenumber]
\setbeamertemplate{navigation symbols}{}

\lstset{basicstyle=\ttfamily\footnotesize,keywordstyle=\color{magenta},tabsize=2,escapeinside=||}

\usetikzlibrary{positioning,overlay-beamer-styles,graphs,fit,shapes}
\tikzset{}

\AtBeginSection[] {
  \begin{frame}{\secname}
    \tableofcontents[currentsection,hideothersubsections,sectionstyle=hide]
  \end{frame}
}

\begin{document}

\frame{\titlepage}

\begin{frame}

Definicje:
\begin{itemize}
  \pause\item
    $p$ - niemalejący ciąg rang pracowników
  \pause\item
    $a_i$ - minimalna liczba pracowników w komitecie zawierającym $p_i$ i reprezentującego $p_1, ..., p_i$;
  \pause\item
    $b_i$ - liczba sposobów na uzyskanie optymalnego komitetu reprezentującego $p_1, ..., p_i$ zawierającego w sobie $p_i$.
\end{itemize}

\pause
Oczywiście $a_0 = 0, b_0 = 1$.

\end{frame}

\begin{frame}

Więcej definicji:
\begin{itemize}
  \pause\item
    $k_i = \max\{ p_j | j < i \land p_j + k < p_i \}$ (największy pracownik \textbf{nie}pokrywany przez $p_i$)
  \pause\item
    $l_i = \min\{ j | p_j + k \ge k_i \}$ (indeks najmniejszego pracownika pokrywającego $k_i$)
  \pause\item
    $r_i = \max\{ j | p_j \le p_i - k \}$ (indeks największego pracownika \textbf{nie}kolidującego z $p_i$)
\end{itemize}

\end{frame}

\begin{frame}

Oczywistym staje się, że
$$a_i = \min\{ a_j | l_i \le j \le r_i \} + 1$$
$$b_i = \sum_{j = l_i}^{r_i} [a_i = a_j + 1] \cdot b_j$$
\pause

$k_i, l_i, r_i$ można łatwo wyznaczyć wyszukiwaniem binarnym (lub w zamortyzowanym czasie liniowym).\\
$a_i, b_i$ można wyznaczyć za pomocą drzewa przedziałowego lub zauważając pewną dodatkową zależność i używając sum prefiksowych, nie będę tego pokazywał, bo jest to bardziej skomplikowane.

Złożoność czasowa: $O(\underbrace{n\log n}_{\text{sortowanie}} + \underbrace{n \log n}_{\text{dynamik}}) = O(n\log n)$.

\end{frame}

\end{document}
